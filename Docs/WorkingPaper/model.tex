%%%%%%%%%%%%%%%%%%%%%%%%%%%%%
% Standard header for working papers
%
% WPHeader.tex
%
%%%%%%%%%%%%%%%%%%%%%%%%%%%%%

\documentclass[11pt]{article}

%%%%%%%%%%%%%%%%%%%%
%% Include general header where common packages are defined
%%%%%%%%%%%%%%%%%%%%



%%%%%%%%%%%%%%%%%%%%%%%%%%
%% TEMPLATES
%%%%%%%%%%%%%%%%%%%%%%%%%%


% Simple Tabular

%\begin{tabular}{ |c|c|c| } 
% \hline
% cell1 & cell2 & cell3 \\ 
% cell4 & cell5 & cell6 \\ 
% cell7 & cell8 & cell9 \\ 
% \hline
%\end{tabular}





%%%%%%%%%%%%%%%%%%%%%%%%%%
%% Packages
%%%%%%%%%%%%%%%%%%%%%%%%%%



% encoding 
\usepackage[utf8]{inputenc}
\usepackage[T1]{fontenc}


% general packages without options
\usepackage{amsmath,amssymb,amsthm,bbm}

% graphics
\usepackage{graphicx,transparent,eso-pic}

% text formatting
\usepackage[document]{ragged2e}
\usepackage{pagecolor,color}
%\usepackage{ulem}
\usepackage{soul}







%%%%%%%%%%%%%%%%%%%%%%%%%%
%% Maths environment
%%%%%%%%%%%%%%%%%%%%%%%%%%

%\newtheorem{theorem}{Theorem}[section]
%\newtheorem{lemma}[theorem]{Lemma}
%\newtheorem{proposition}[theorem]{Proposition}
%\newtheorem{corollary}[theorem]{Corollary}

%\newenvironment{proof}[1][Proof]{\begin{trivlist}
%\item[\hskip \labelsep {\bfseries #1}]}{\end{trivlist}}
%\newenvironment{definition}[1][Definition]{\begin{trivlist}
%\item[\hskip \labelsep {\bfseries #1}]}{\end{trivlist}}
%\newenvironment{example}[1][Example]{\begin{trivlist}
%\item[\hskip \labelsep {\bfseries #1}]}{\end{trivlist}}
%\newenvironment{remark}[1][Remark]{\begin{trivlist}
%\item[\hskip \labelsep {\bfseries #1}]}{\end{trivlist}}

%\newcommand{\qed}{\nobreak \ifvmode \relax \else
%      \ifdim\lastskip<1.5em \hskip-\lastskip
%      \hskip1.5em plus0em minus0.5em \fi \nobreak
%      \vrule height0.75em width0.5em depth0.25em\fi}



%%%%%%%%%%%%%%%%%%%%
%% Idem general commands
%%%%%%%%%%%%%%%%%%%%
%% Commands

\newcommand{\noun}[1]{\textsc{#1}}


%% Math

% Operators
\DeclareMathOperator{\Cov}{Cov}
\DeclareMathOperator{\Var}{Var}
\DeclareMathOperator{\E}{\mathbb{E}}
\DeclareMathOperator{\Proba}{\mathbb{P}}

\newcommand{\Covb}[2]{\ensuremath{\Cov\!\left[#1,#2\right]}}
\newcommand{\Eb}[1]{\ensuremath{\E\!\left[#1\right]}}
\newcommand{\Pb}[1]{\ensuremath{\Proba\!\left[#1\right]}}
\newcommand{\Varb}[1]{\ensuremath{\Var\!\left[#1\right]}}

% norm
\newcommand{\norm}[1]{\left\lVert #1 \right\rVert}



% argmin
\DeclareMathOperator*{\argmin}{\arg\!\min}


% amsthm environments
\newtheorem{definition}{Definition}
\newtheorem{proposition}{Proposition}
\newtheorem{assumption}{Assumption}

%% graphics

% renew graphics command for relative path providment only ?
%\renewcommand{\includegraphics[]{}}





%\renewcommand{\abstractname}{}


\newcommand{\cinzia}[1]{\textcolor{red}{\textbf{\textit{#1}}}}

% comments and responses
\usepackage{xparse}
\DeclareDocumentCommand{\comment}{m o o o o}
{%
    \textcolor{red}{#1}
    \IfValueT{#2}{\textcolor{blue}{#2}}
    \IfValueT{#3}{\textcolor{ForestGreen}{#3}}
    \IfValueT{#4}{\textcolor{red!50!blue}{#4}}
    \IfValueT{#5}{\textcolor{Aquamarine}{#5}}
}


% todo
\newcommand{\todo}[1]{\textcolor{red!50!blue}{\textbf{\textit{#1}}}}



%\usepackage[UTF8]{ctex}



% geometry
\usepackage[margin=2cm]{geometry}

% layout : use fancyhdr package
\usepackage{fancyhdr}
\pagestyle{fancy}

\makeatletter

\renewcommand{\headrulewidth}{0.4pt}
\renewcommand{\footrulewidth}{0.4pt}
\fancyhead[RO,RE]{\textit{Working Paper}}
\fancyhead[LO,LE]{G{\'e}ographie-Cit{\'e}s}
\fancyfoot[RO,RE] {\thepage}
\fancyfoot[LO,LE] {\noun{C. Losavio and J. Raimbault}}
\fancyfoot[CO,CE] {}

\makeatother


%%%%%%%%%%%%%%%%%%%%%
%% Begin doc
%%%%%%%%%%%%%%%%%%%%%

\begin{document}







\title{Agent-based Modeling of Migrant Workers Residential Dynamics within a Mega-city Region: the Case of Zhuijiang Delta, China
%Challenging Preconceived Ideas with an Agent-based Model
\bigskip\\
\textit{Working Paper}
}
\author{\noun{Cinzia Losavio} and \noun{Juste Raimbault}\\
UMR CNRS 8504 Géographie-cités
}
\date{}


\maketitle

\justify


\begin{abstract}
Rural-urban migrant workers have played in the last decades an increasing role for China's economy, raising the attention on associated socio-economical issues. The importance of their economic diversity is however poorly known, as the strategy of the State towards these new classes. We propose to investigate both by means of agent-based modeling, at the scale of Pearl River Delta Mega-city Region. Our model couples patch-level population and economic structure, which evolution is determined in a top-down fashion from meso-dynamics following simple growth rules, with microscopic residential discrete choices dynamics of migrant workers. This two level structure is essential to control on city system structure. The model is explored on synthetic data and internally validated for statistical consistence and path-dependancy of temporal trajectories. The application to Zhuijiang Delta region allows first to test the sensitivity of outcomes to population economic structure, for which spread wealth distributions yield more complex trajectories, and secondly to compare the impact of different State policies on migration control.

\bigskip

\noindent\textbf{Keywords : } \textit{Mingong ; Residential Dynamics ; Agent-based Modeling ; Zhuijiang Delta Mega-city Region}
\end{abstract}




%%%%%%%%%%%%%%%%%%%%
\section{Introduction}



\paragraph{Modeling Rural-urban migrations in China}

Existing works in rural-urban migration modeling in China are mainly econometric studies, relying on census or on survey data. \cite{zhang2013measuring} estimate discrete choice models to study the tradeoff between migration distance and earning difference. \cite{fan2005modeling} shows that gravity-based models can explain well inter-provincial migratory patterns, implying an underlying strong dominant aggregation processes. The positive association between wage gap and migration rates was obtained from time-series analysis in~\cite{zhang2003rural}.

\paragraph{Towards an agent-based modeling approach}

To the best of our knowledge, there was no attempt in the literature before to focus on China's migration issues from an agent-based perspective. The case of Mexico was tackled by~\cite{de2007netlogo}, but in the particular case of a border-town, and underlying processes are furthermore fundamentally different. \textit{TODO : find more literature}

Following a logic of \emph{Pattern-oriented modeling}~\cite{grimm2005pattern}, combined with recent advances in multi-modeling~\cite{cottineau2016back}, one can use agent-based models as powerful to test qualitative hypothesis, with a reasonable need for empirical data through toy-models or hybrid models.



%%%%%%%%%%%%%%%%%%%%
\section{Model}

\subsection{Rationale}

We choose to focus on particular processes and stylized facts to include in the agent-based models, in order to test some hypothesis formulated after qualitative research done in~\cite{losavio2016analyser}. More precisely, a recent shift in socio-economic structure of migrating population was observed, including a rise of middle-income migrants and a relativisation of the role of \emph{Hukou} in migration dynamics. The core of the model is thus centered on the exploration of the impact of a varying population economic structure for migrants on system dynamics, and the influence of government migration policies.

\paragraph{Scale} As shown by~\cite{chan2012migration}, recent migration dynamics feature a high asymmetry from central region to coastal economically dynamic province. At a macroscopic scale, explanatory variables are relatively well understood and gravity-based geographical models have a reasonable explanatory power~\cite{fan2005modeling}. However, at a regional scale, migration dynamics are also highly present and present more complex patterns. The scale of the model is therefore a regional scale, in the spirit of a \emph{Mega-city Region} \cite{hall2006polycentric}, in which urban dynamics are highly complex. A relevant case study to apply the model will be the Pearl River Delta Region in Guangdong.

\paragraph{Ontology} At a mesoscopic scale, i.e. for cities within the MCR, growth can be reasonably assumed independent from migrants movements : they follow larger economic urban processes. Conditionally to such population and economic context, migration dynamics occur at the microscopic scale. We postulate a simple utility-maximization process.

\subsection{Model Description}

\paragraph{Setup}

The world consists in a lattice of $1 \leq i \leq N$ cells, characterized by their population $P_i(t)$ and an economic structure $E_i^{(c)}(t)$ which consists in abstract variables representing potential number of jobs stratified by socio-economic classes $c$. The associated effective number of workers is denoted by $W_i^{(c)}(t)$. For the sake of simplicity, we assume a discrete number of classes. At initial time, the variables are initialized either following a synthetic data generation process (see below), or from real geographical data (abstracted and simplified to fit our context). Cells are grouped into $1\leq k\leq K$ administrative cities that corresponding to the various centers of the MCR, on which aggregated population $\tilde{P}_k(t)$ and corresponding economic variables can be computed.

\paragraph{Agents}

An agent is a household of migrants, whose residence and job are located in cells (that can be different). Socio-economic structure of the population is captured by the distribution of wealth $g(w)$, which are then stratified into categories. At a given time, the utility difference between not moving and moving to cell $j$ from cell $i$, for a category $c$ is given by

\[
\Delta U_{i,j}^{(c)}(t) = \frac{Z_j^{(c)}- Z_i^{(c)}}{Z_0} + \frac{C_i^{(c)}- C_j^{(c)}}{C_0} - u_i^{(c)} - h_j^{(c)}
\]

where $Z_i^{(c)}$ is generalized accessibility given by $Z_i^{(c)} = P_i \cdot \sum_k \left[E_k^{(c)}-W_k^{(c)}\right]\cdot \exp{\left(\frac{-d_{ij}}{d_0}\right)}$, with $d_{ij}$ effective travel distance (in public transportation ; \textit{point to be clarified : for higher classes, car may be an option}) and $d_0$ commuting characteristic distance ; $C_i^{(c)}$ is the cost of life which is a function of cell and city variables, that will be taken as $C_i^{(c)} \propto P_i^{\alpha_0}\cdot  \tilde{P}_i^{\alpha_1}\cdot$ ; $u_i^{(c)}$ a baseline aversion to move and $h_j^{(c)}$ an exogenous variable corresponding to regulation policies ; $Z_0$ and $C_0$ dimensioning parameters.

\paragraph{Temporal Evolution}

At each time step, the system evolves sequentially according to the following rules :

\begin{itemize}
\item Cities mesoscopic variables $\tilde{P}_k(t)$ and $\tilde{E}_k^{c}(t)$ are deterministically updated, following the very simple assumption of the expectancy of a Gibrat's law, and a scaling law to determine economic variables from population.
\item Patches variables are updated conditionally to the new aggregated values. \textit{Using a simple urban growth model of aggregation, deterministic for population and a bit more random for jobs ; still to be clarified}.
\item A number of new migrants, proportional to Gibrat growth rate, enter the region and settle randomly, by first choosing a random available job location in their category, and then a residence nearby with a probability function of population density and exponentially decreasing with distance to job, as $\exp{\left(\frac{-d_{ij}}{d_0}\right)}$. This procedure is also used for initial setup.
\item Migration occur following a discrete choice dynamics : the probability to move to cell $j$ is given by

\[
\Pb{i\rightarrow j | c} = \frac{\exp{\left(\beta\cdot U_j^{(c)}\right)}}{\sum_k \exp{\left(\beta\cdot U_k^{(c)}\right)}+\exp{\left(U_{stay,i}^{(c)}\right)}}
\]

what simplifies into a reduced form, with $\beta' = \frac{\beta}{Z_0}$, $\gamma = \frac{Z_0}{C_0}$ and $\tilde{u},\tilde{h}$ accordingly rescaled variables, using the above utility expression :

\[
\Pb{i\rightarrow j | c} = \frac{\exp{\left(\beta'\cdot \left[\Delta Z_{i,j}^{(c)} - \Delta C_{i,j}^{(c)} - \tilde{u}_i^{(c)} - \tilde{h}_j^{(c)} \right] \right)}}{1 + \sum_k {\exp{\left(\beta'\cdot \left[\Delta Z_{i,k}^{(c)} - \Delta C_{i,k}^{(c)} - \tilde{h}_k^{(c)} \right] \right)} - N\cdot \tilde{u}_i^{(c)}}}
\]

Residential movement is drawn randomly according to these probabilities, and jobs are chosen around new residence following an exponentially decreasing probability.

\end{itemize}


\paragraph{Indicators}

\textit{TBW}

\paragraph{Synthetic Data Generation}

\textit{TBW}



%%%%%%%%%%%%%%%%%%%%
\section{Results}

\paragraph{Model Validation}

\begin{itemize}
\item Internal validation : statistical consistence ; system trajectories ; path-dependency.
\item External validation : stylized facts from synthetic data exploration ?
\end{itemize}


\paragraph{Application}

\textit{Stylize Pearl River Delta configuration}

\paragraph{Experience Plan} \textit{concrete qualitative questions that can be asked to the model, e.g. :
\begin{itemize}
\item what is the impact of varying wealth distribution shape and width on system dynamics ?
\item what is the impact of various spatial distribution of $h_j^{(c)}$, i.e. different government policies ?
\item \ldots
\end{itemize}
}



%%%%%%%%%%%%%%%%%%%%
%% Biblio
%%%%%%%%%%%%%%%%%%%%

\bibliographystyle{apalike}
\bibliography{/Users/Juste/Documents/ComplexSystems/MigrationDynamics/Biblio/mingong}


\end{document}
